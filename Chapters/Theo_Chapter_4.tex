\chapter{Relativitätstheorie und Elektrodynamik}

Ziel dieses Kapitels ist es, die Maxwellgleichungen relativistisch-kovariant darstellen zu können. Hierzu benötigen wir aber zunächst eine kurze Wiederholung der Formalismen der Relativitätstheorie.

\section{Spezielle Relativitätstheorie (Wiederholung)}

\subsubsection{Inertialsysteme:}

Bezugssysteme, in denen sich ein Kräftefreier Körper geradlinig und gleichförmig bewegt.

\subsection{Newtonsche Mechanik und Galileitransformation (Exkurs)}

\subsubsection{Newtonsche Mechanik}

in der \textbf{Newtonschen Mechanik} gilt das Galileische Relativitätsprinzip:\\
Alle IS sind gleichwertig, d.h. physikalische Gesetze haben in allen IS die gleiche Form.\\[5pt]
Der Übergang zwischen IS verläuft mittels der \textbf{Galileitransformation}:
%
%
%
% T5
%
%
%
$ \vec{v} = v \vec{e}_x \qquad \vec{r}' = \vec{r} - \vec{v}t $\\
Transformationskoordinaten:
\begin{align*}
x' = x - vt \quad y' = y \quad z' = z \quad t' = t
\end{align*}
Transformation von Geschwindigkeiten
%
%
%
% T6
%
%
%
\begin{equation*}
u' \defeq \frac{x'}{t'} = \frac{x - vt}{t} = \frac{x}{t} - v = u - v
\end{equation*}
\begin{equation*}
u = u' + v
\end{equation*}
Geschwindigkeitsaddition ist also linear.\\[5pt]
\textbf{Exp:} Lichtgeschwindigkeit $ c $ ist in allen Systemen gleich \LARGE \lightning \normalsize

\subsubsection{Einsteinsches Relativitätsprinzip}

Grundlagen:
\begin{enumerate}[1)]
	\item Alle IS sind gleichwertig
	\item Die Lichtgeschwindigkeit $ c $ ist in allen IS gleich
\end{enumerate}

\subsection{Lorentztransformation und relativistische Notation}

%
%
%
% T7
%
%
%
Ereignis in $ S $ bei $ t, x, y, z $ hat in $ S' $ die Koordinaten $ t', x', y', z' $\\
Der Zusammenhang ist gegeben durch die Lorentztransformation.

\subsubsection{Lorentransformation}

\begin{equation*}
t' = \gamma \left( - \frac{\beta}{c} x + t\right) \quad x' = \gamma \left(x - vt\right) \quad y' = y \quad z' = z
\end{equation*}
\begin{equation*}
\beta = \frac{v}{c} \qquad \gamma = \frac{1}{\sqrt{1 - \beta^2}}
\end{equation*}
Grenzfall: $ \left| \frac{v}{c} \right| \ll 1 : \quad \beta \to 0 , \gamma \to 1 $\\
$ \Rightarrow \quad t' = t \quad x' = c - vt \quad y' = y \quad z' = z $

\subsubsection{relativistische Notation}

$ t, x, y, z : $ Ereignis im \textbf{Minkoswki-Raum}\\[5pt]
\textbf{Vierervektor}
\begin{equation*}
\left(x^\mu\right): \quad x^\mu \quad \mu = 0, 1, 2, 3
\end{equation*}
\begin{equation*}
x^0 = ct \quad x^1 = x \quad x^2 = y \quad x^3 = z \qquad (x^\mu) = \begin{pmatrix}
ct \\ x \\ y \\ z
\end{pmatrix}
\end{equation*}
Darstellung: Raum-Zeit-Diagramme oder auch Minkowski-Diagramme:
%
%
%
% T8
%
%
%
Abstand zweier Ereignisse
\begin{equation*}
(x_A^\mu) = (x_A^0 , x_A^1, x_A^2, x_A^3) = (ct_A, x_A, y_A, z_A)
\end{equation*}
\begin{equation*}
(x_B^\mu) = (x_B^0 , x_B^1, x_B^2, x_B^3) = (ct_B, x_B, y_B, z_B)
\end{equation*}
Abstand:
\begin{align*}
(\Delta s)^2 &\defeq (x_A^0 - a_B^0)^2 - (x_A^1 - x_B^1)^2 - (x_A^2 - x_B^2)^2 - (x_A^3 - x_B^3)^2\\
&= c^2 (t_A - t_B)^2 - (x_A - x_B)^2 - (y_A - y_B)^2 - (z_A - z_B)^2
\end{align*}
Wegelement $ \dd s $:
\begin{align*}
(\dd s)^2 &= c^2 \dd t^2 - \dd x^2 - \dd y ^2 - \dd z^2\\
&= (\dd x^0)^2 - (\dd x^1)^2 - (\dd x^2)^2 - (\dd x^3)^2\\
&= \sum_{\mu \nu} g_{\mu \nu} \dd x^\mu \dd x^\nu
\end{align*}
mit $ g_{\mu \nu} $ dem metrischen Tensor: $ g_{\mu \nu} = \begin{pmatrix}
1 & 0 & 0 & 0 \\
0 & -1 & 0 & 0 \\
0 & 0 & -1 & 0 \\
0 & 0 & 0 & -1
\end{pmatrix} $
Lorentztransformation: $ S \to S' $
\begin{equation*}
x^\mu \to x'^{\mu} \to \begin{pmatrix}
ct' \\ x' \\ y' \\ z'
\end{pmatrix} = \ob{\begin{pmatrix}
\gamma & - \beta \gamma & 0 & 0 \\
- \beta \gamma & \gamma & 0 & 0 \\
0 & 0 & 1 & 0 \\
0 & 0 & 0 & 1
\end{pmatrix}}^{\eqdef \Lambda^{\mu}_{\ \nu}} \begin{pmatrix}
ct \\ x \\ y \\ z
\end{pmatrix}
\end{equation*}
\textbf{Einsteinsche Summen-Konvention}
\begin{equation*}
x'^{\mu} = \sum_{\nu} \Lambda^{\mu}_{\ \nu} x^{\nu} \defeq \Lambda^{\mu}_{\ \nu} x^{\nu}
\end{equation*}
\lcom{Bei gleichen Indices, einer oben, einer unten, wird die Summe weggelassen es wird dann über diesen Index aufsummiert.}

\subsection{Skalare, Vektoren, Matrizen, Tensoren in der vierdimansionalen Raum-Zeit}

\subsubsection{Skalare}

Größen, die invariant sind unter Lorentztransformation (LT)\\[5pt]
\emph{Beispiele:}
\begin{enumerate}[i)]
	\item Abstand: $ \Delta s, \dd s $
	\item Eigenzeit: $ \dd \tau = \sqrt{1 - \left(\frac{\vec{v}(t)}{c}\right)^2} \dd t $ 
	\item $ c, m_0, q $
\end{enumerate}

\subsubsection{Vierervektoren}

Ortsvektor: $ x'^{\mu} = \Lambda^{\mu}_{\nu} x^{\nu} $\\[5pt]
Ein Vierervektor $ b $ ist eine vierkomponentige Größe $ (b^{\mu}) $, welche sich bei LT wie die Komponenten des Ortsvektors transformiert:
\begin{equation*}
b'^{\mu} = \Lambda^{\mu}_{\ \nu} b^{\nu}
\end{equation*}
\emph{Beispiele:}
\begin{enumerate}[i)]
	\item Ortsvektor
	\item Vierergeschwindigkeit $ u = (u^{\mu}) $\\
	\lcom{Wurde eingeführt, da sich die normale Geschwindigkeit nicht wie der Ortsvektor transformiert.}
	\begin{equation*}
	u^{\mu} \defeq \prt{x^{\mu}}{\tau}
	\end{equation*}
	\begin{align*}
	\dd x^{\mu} &= (\dd x^0, \dd x^1, \dd x^2, \dd x^3)\\
	&= (c \dd t, \dd x, \dd y, \dd z)
	\end{align*}
	$ \dd \tau' = \dd \tau $
	\begin{equation*}
	\dd x'^{\mu} = \Lambda^{\mu}_{\nu} \dd x^{\nu} \qquad u'^{\mu} = \prt{x'^{\mu}}{\tau'} = \Lambda^{\mu}_{\nu} u^{\nu}
	\end{equation*}
	\begin{equation*}
	u^{0} = \prd{x^{0}}{\tau} = \gamma \prd{}{t} ct = \gamma c
	\end{equation*}
	\begin{equation*}
	u^{1} = \prd{x^1}{\tau} = \gamma \prd{x}{t} = \gamma v_x
	\end{equation*}
	\begin{equation*}
	(u^{\mu}) = \gamma \begin{pmatrix}
	c \\ \vec{v}
	\end{pmatrix}
	\end{equation*}
	\begin{equation*}
	x'^{\mu} = \Lambda^{\mu}_{\ \nu} x^{\nu} \qquad x'_{\mu} = \ol{\Lambda}_{\mu}^{\ \nu} x_\nu 
	\end{equation*}
\end{enumerate}
,,Skalarprodukt`` zweier Vierervektoren
\begin{equation*}
(a^{\mu}), (b^{\mu}) : \quad a \cdot b \defeq a^0 b^0 - a^1 b^1 - a^2 b^2 - a^3 b^3
\end{equation*}
\begin{equation*}
\rightarrow \quad (\Delta s)^2 = (x_A - x_B) \cdot (x_A - x_B)
\end{equation*}

\subsubsection{Kovariante und Kontravariante Vektorkomponenten}

\begin{equation*}
(a^{\mu}) = (a^0, a^1, a^2, a^3)
\end{equation*}
\begin{equation*}
(a_{\mu}) \defeq (a^0, -a^1, -a^2, -a^3)
\end{equation*}
$ a^{\mu} : $ Kontravariante Komponenten\\
$ a_{\mu} : $ Kovariante Komponenten
\begin{equation*}
a \cdot b = \sum_{\mu} a^{\mu} b_{\mu} = a^{\mu} b_{\mu}
\end{equation*}
$$ g_{\mu \nu} : \qquad x_{\mu} = g_{\mu \nu} x^{\nu} \qquad x^{\mu} = g^{\mu \nu} x_{\nu} $$
Dies nennt man auch das Herauf- oder Herunterziehen der Indizes.
\begin{align*}
x_1 &= g_{1 \nu} x^\nu\\
&= \cancel{g_{10}}x^0 + g_{11} x^2 + \cancel{g_{12}}x^2 + \cancel{g_{13}}x^3\\
&= - x^1
\end{align*}

\subsubsection{Tensoren}

Größe mit oberen und/oder unteren Indizes, wobei sich jeder obere Index Kontravariant und jeder untere Index Kovariant transformiert.\\[5pt]
\emph{Beispiel:} Tensor 3. Stufe
$$ T^{\alpha \beta}_{\ \ \ \gamma} : \quad T'^{\alpha \beta}_{\ \ \ \: \gamma} = \Lambda^{\alpha}_{\mu} \Lambda^{\beta}_{\nu} \ol{\Lambda}_{\gamma} ^{\ \sigma} T^{\mu \nu}_{\ \ \ \sigma} $$
metrischer Tensor: $ g_{\mu \nu} $, Feldstärketensor $ F^{\mu \nu} $
%
%
%
% check wecher index zuerst kommt
%
%
%

% Vorlesung 4.2.

\subsubsection{Vierervektoren}

\begin{equation*}
(a^{\mu}): \quad a'^{\mu} = \Lambda^{\mu}_{\ \nu} a^{\nu} \qquad \tx{kontravariant}
\end{equation*}
$ \mu, \nu = 0, 1, 2, 3 $
%
%
%
% T1
%
%
%
Boosts:
\begin{equation*}
\Lambda^{\mu}_{\ \nu} = \begin{pmatrix}
\gamma & - \beta \gamma & 0 & 0 \\
- \beta \gamma & \gamma & 0 & 0 \\
0 & 0 & 1 & 0 \\
0 & 0 & 0 & 1
\end{pmatrix} \qquad \gamma = \frac{1}{\sqrt{1 - \frac{v^2}{c^2}}} \qquad \beta = \frac{v}{c}
\end{equation*}
\begin{equation*}
(x^{\mu}) = \left(ct, x, y, z\right)
\end{equation*}
\begin{equation*}
(a_{\mu}) : \quad \tx{kovariant} \qquad a_{\mu} = g_{\mu \nu} a ^{\nu}
\end{equation*}
\begin{equation*}
g_{\mu \nu} = \begin{pmatrix}
1 & 0 & 0 & 0 \\
0 & -1 & 0 & 0 \\
0 & 0 & -1 & 0 \\
0 & 0 & 0 & -1
\end{pmatrix} = g^{\mu \nu}
\end{equation*}
\begin{equation*}
(x_{\mu}) = (ct, -x, -y, -z)
\end{equation*}
\begin{align*}
a'_{\mu} &= g_{\mu \nu} a'^{\nu} = g_{\mu \nu} \Lambda^{\nu} _{\ \alpha} a ^{\alpha}\\
&= \ub{g_{\mu \nu} \Lambda^{\nu}_{\ \alpha} g^{\alpha \beta}}_{\ol{\Lambda}_{\mu} ^{\ \beta}} a_{\beta}
\end{align*}
\begin{equation*}
\Rightarrow \quad a'^{\mu} = \ol{\Lambda}_{mu}^{\ \beta} a_{\beta}
\end{equation*}
\begin{equation*}
\ol{\Lambda} = \left(\Lambda^{-1}\right)^{\top}
\end{equation*}
Boosts: $ \Lambda(\vec{v}) $:
\begin{align*}
\left(\Lambda(\vec{v})\right)^{-1} &= \Lambda(-\vec{v}) \qquad \Lambda^{\top} = \Lambda
\end{align*}
\begin{equation*}
\ol{\Lambda} = \Lambda(- \vec{v})
\end{equation*}
\emph{Beispiel:} partielle Ableitung (Gradienten);
\begin{equation*}
\partial_{\mu} \defeq = \prt{}{x^{\mu}} \qquad \tx{transformiert sich wie kovariante Komponente}
\end{equation*}
\begin{align*}
x'^{\mu} &= \Lambda ^{\mu} _{\ \nu} x^{\nu} \\
\Rightarrow \quad x^{\nu} &= \ub{\left(\Lambda^{-1}\right)^{\nu}_{\ \mu}}_{\ol{\Lambda}_{\mu} ^{\ \nu}} x'^{\mu} = \ol{\Lambda}_{\mu} ^{\ \nu} x'^{\mu}
\end{align*}
\begin{equation*}
\Rightarrow \quad \prt{x^{\nu}}{x'^{\mu}} = \ol{\Lambda} _{\mu} ^{ \ \nu}
\end{equation*}
\begin{align*}
\partial '_{\mu} &= \prt{}{x'^{\mu}} = \sum_{\nu} \prt{x^{\nu}}{x'^{\mu}} \prt{}{x^{\nu}} \\
&= \sum_{\nu} \Lambda_{\mu}^{\ \nu} \prt{}{x^{\nu}} = \ol{\Lambda}_{\mu}^{\ \nu} \partial_{\nu}
\end{align*}
\begin{equation*}
\partial ^{\mu} \defeq \prt{}{x_{\mu}} \qquad \partial '^{\mu} = \Lambda^{\mu} _{\ \nu} \partial ^{\nu}
\end{equation*}

\section{Relativistisch kovariante Formulierung der Elektrodynamik}

\begin{enumerate}[1)]
	\item \textbf{Ladungs- und Stromdichte} $ \rho, \vec{j} $
	\begin{equation*}
	\rho , \vec{j} \to \tx{ Vierervektor}: \quad (j^{\mu}) \defeq ( c \rho, \vec{j}) = ( c \rho, j_x, j_y, j_z) 
	\end{equation*}
	Ladung $ q $: \textbf{Lorentz-Skalar}: $ \quad q' = q $\\
	\textbf{Ladungserhaltung}: \textbf{Kontinuitätsgleichung} gilt: $ \prt{\rho}{t} + \vabla \cdot \vec{j} = 0 $
	\begin{align*}
	0 &= \prt{\rho}{t} + \vabla \cdot \vec{j} = \ub{\prt{}{(ct)} (c \rho)}_{= \prt{}{x^{0}} j^{0}} + \sum_{\mu=1}^{3} \prt{}{x^{\mu}} j^{\mu} \qquad (x^{\mu}) = (ct, x, y, z)\\
	&= \sum_{\mu=1}^{3} \prt{}{x^{\mu}} j^{\mu} = \partial_{\mu} j^{\mu}
	\end{align*}
	Kontinuitätsgleichung in relativistischer Schreibweise
	\begin{equation*}
	\rmbox{\partial _{\mu} j^{\mu} = 0}
	\end{equation*}
	\begin{equation*}
	\partial_{\mu} j^{\mu} = 0 = \partial'_{\mu} j'^{\mu}
	\end{equation*}
	es wird also über $ \mu $ aufsummiert: $ a_{\mu} b_{\mu} $\\[10pt]
	Lorentztransformation: $ S \to S' $\\
	$ j'^{\mu} = \Lambda^{\mu}_{\ \nu} j^{\nu} $\\
	\emph{Beispiel:}
	%
	%
	%
	% T2
	%
	%
	%
	\begin{equation*}
	\left(\Lambda ^{\mu}_{\ \nu}\right) = \begin{pmatrix}
	\gamma & - \beta \gamma & 0 & 0 \\
	- \beta \gamma & \gamma & 0 & 0 \\
	0 & 0 & 1 & 0 \\
	0 & 0 & 0 & 1
	\end{pmatrix}
	\end{equation*}
	\begin{align*}
	j'^{0} &= c\rho ' = \gamma c \rho - \beta \gamma j_x \\
	j'^{1} &= j'_{x} = - \beta \gamma c \rho + \gamma j_x \\
	j'^{2} &= j'_{y} = j_y \\
	j'^{3} &= j'_{z} = j_z \\
	\end{align*}
	Annahme: in $ S $ ruhende Ladung $ q $ im Volumen $ \Delta V: $ $ \rho = \frac{q}{\Delta V} , \vec{j} = 0 $\\[5pt]
	in $ S' $:
	\begin{equation*}
	\rho' = \gamma \rho = \gamma \frac{q}{\Delta V} = \frac{q}{\frac{\Delta V}{\gamma}} = \frac{q}{\Delta V'} \ge \rho
	\end{equation*}
	Das Volumen $ \Delta V' $ ist kleiner (Längenkontraktion), daher ist die Ladungsdichte in $ S' $ größer als in $ S $.\\
	Längenkontraktion: $ \Delta V' = \frac{\Delta V}{\gamma} $
	\begin{align*}
	j'_x &= - \beta \gamma c \rho = - \beta c \rho'\\
	&= - v \rho'
	\end{align*}
	\lcom{Die $ y $- und $ z $-Komponenten sind gleich null, da sich $ S' $ nur in $ x $-Richtung von $ S $ wegbewegt.}
	\item \textbf{Viererpotential}
	\begin{equation*}
	\Phi, \vec{A} \to (A^{\mu}) = (\frac{1}{c} \Phi, A_x, A_y, A_z)
	\end{equation*}
	\lcom{Das $ \frac{1}{c} $ kommt aus dem SI-Einheitensystem, sodass die Einheiten der Vektorkomponenten übereinstimmen. (Im CGS-System gibt es diesen Faktor nicht)}\\[5pt]
	\textbf{Lorenzeichung:} $ \vabla \cdot \vec{A} + \frac{1}{c^2} \prt{\Phi}{t} = 0 $
	\begin{align*}
	\rightarrow \quad 0 &= \prt{}{(ct)} \left(\frac{1}{c} \Phi\right) + \vabla \cdot \vec{A} = \sum_{\mu} \prt{}{x^{\mu}} A^{\mu} = \partial _{\mu} A^{\mu}
	\end{align*}
	\begin{equation*}
	\rmbox{\partial_{\mu} A_{\mu} = 0}
	\end{equation*}
	\lcom{Da Skalarprodukte in der Relativitätstheorie Lorentzinvariant sind, sieht man hier, dass auch die Lorenzeichugn in der Relativitätstheorie Lorentzinvariant ist.\\
	Dies gilt nicht für die Coulomb-Eichung.}\\[5pt]
	Coulomb-Eichung: $ \vabla \cdot \vec{A} = 0 $\\[10pt]
	Die Maxwell-Geichungen führen aus den Potential-Gleichungen auf:
	\begin{align*}
	\Rightarrow \quad \left(\Delta - \frac{1}{c^2} \prt{^2}{t^2}\right) \Phi &= - \frac{1}{\epsilon_0} \rho \\
	\left(\Delta - \frac{1}{c^2} \prt{^2}{t^2}\right) \vec{A} &= - \mu \vec{j}
	\end{align*}
	Viererlaplaceoperator:
	\begin{equation*}
	\partial_{\nu} \partial^{\nu} = \sum_{\nu} \prt{}{x^{\nu}} \prt{}{x_{\nu}} = \frac{1}{c^2} \prt{^2}{t^2} - \prt{^2}{x^2} - \prt{^2}{y^2} - \prt{^2}{z^2} \eqdef \square \quad \tx{\textbf{d' Alambert Operator}}
	\end{equation*}
	\begin{equation*}
	\square A^{\nu} = \mu_0 j^{\nu} \qquad \nu = 0, 1, 2, 3
	\end{equation*}
	$ \nu = 0 $: $ A^{0} = \frac{1}{c} \Phi \quad j^{0} = c \rho $
	\begin{equation*}
	\square \frac{1}{c} \Phi = \mu_0 c \rho
	\end{equation*}
	\begin{equation*}
	\square \Phi = \mu_0 c^2 \rho = \frac{1}{\epsilon_0} \rho
	\end{equation*}
	\emph{Beispiel:} Potentiale einer gleichförmig bewegten Ladung $ q $
	%
	%
	%
	% T3
	%
	%
	%
	$ q $ ruhe um Ursprung von $ S' $\\[5pt]
	Potentiale in $ S' $:
	\begin{equation*}
	\Phi'(\vec{r}') = \frac{1}{4 \pi \epsilon_0} \frac{q}{|\vec{r}|} \quad \vec{A}'(\vec{r}') = 0
	\end{equation*}
	\begin{equation*}
	\Rightarrow \quad \left(A'^{\nu} (x')\right) = \left(\frac{1}{c} \Phi', \vec{A}'\right) = \left(\frac{1}{c} \frac{1}{4 \pi \epsilon_0} \frac{q}{|\vec{r}'|}, 0, 0, 0\right)
	\end{equation*}
	Transformation von $ S' $  ins Laborsystem $ S $ (LT mit $ - \vec{v} $):\\[5pt]
	LT: $ \vec{v} = -v \vec{e}_x $
	\begin{equation*}
	\Lambda = \begin{pmatrix}
	\gamma & \beta \gamma & 0 & 0 \\
	\beta \gamma & \gamma & 0 & 0 \\
	0 & 0 & 1 & 0 \\
	0 & 0 & 0 & 1
	\end{pmatrix}
	\end{equation*}
	\begin{equation*}
	A^{\nu} (x) = \Lambda^{\nu}_{\ \alpha} A'^{\alpha} (x')
	\end{equation*}
	\begin{align*}
	A^{0} &= \gamma A'^{0} + \beta \gamma A'^{1} = \frac{\gamma}{c} \frac{1}{4 \pi \epsilon_0} \frac{q}{\vec{r}'}\\
	&= \frac{1}{c} \Phi
	\end{align*}
	\begin{equation*}
	\rightarrow \quad \rmbox{\Phi (\vec{r}) = \frac{\gamma q}{4 \pi \epsilon_0 r'}}
	\end{equation*}
	\begin{align*}
	A^{1} &= \beta \gamma A'^{0} + \gamma A'^{1}\\
	&= \frac{\beta \gamma}{c} \frac{q}{4 \pi \epsilon_0 r'} = A_x\\
	A_{x }&=  \frac{\beta \gamma}{c} \frac{q}{4 \pi \epsilon_0 r'}
	\end{align*}
	\begin{align*}
	A^{2} &= A'^{2} = 0\\
	A^{3} &= A'^{3} = 0
	\end{align*}
	mit $ \beta = \frac{v}{c} $
	\begin{equation*}
	\Rightarrow \quad \rmbox{\vec{A}(\vec{r}) = \frac{\mu_0}{4 \pi} \frac{v \gamma q}{r'} \vec{e}x}
	\end{equation*}
	Umrechnung: $ x' \to x $
	\begin{equation*}
	\Lambda^{-1} = \begin{pmatrix}
	\gamma & - \beta \gamma & 0 & 0 \\
	- \beta \gamma & \gamma & 0 & 0 \\
	0 & 0 & 1 & 0 \\
	0 & 0 & 0 & 1
	\end{pmatrix}
	\end{equation*}
	\begin{equation*}
	(x'^{\nu}) = ( ct', x', y', z') = (\gamma x^{0} - \beta \gamma x^{1}, - \beta \gamma x^{0} + \gamma x^{1}, x^{2}, x^{3})
	\end{equation*}
	\begin{align*}
	r'^{2} &= x'^2 + y'^2 + z'^2\\
	&= (\gamma x - \beta \gamma ct)^2 + y^2 + z^2\\
	&= \gamma^2 (x - vt)^2 + y^2 + z^2
	\end{align*}
	\begin{align*}
	\Phi(\vec{r}) &= \frac{1}{4 \pi \epsilon_0}\frac{q}{\sqrt{(x - vt)^2 + (1 - \frac{v^2}{c^2})(y^2 + z^2)}} \\
	\vec{A}(\vec{r}) &= \ \frac{\mu_0}{4 \pi} \ \: \frac{v q}{\sqrt{(x - vt)^2 + (1 - \frac{v^2}{c^2})(y^2 + z^2)}}
	\end{align*}
	Zuvor: gleiches Ergebnis in anderer Schreibweise:
	\begin{equation*}
	\Phi(\vec{r}) = = \frac{1}{4 \pi \epsilon_0} \frac{q}{|\vec{r} - \vec{R}(t)| \sqrt{1 - \frac{v^2}{c^2} \sin^2 \alpha}}
	\end{equation*}
	%
	%
	%
	% T4
	%
	%
	%
	
	%Vorlesung 7.2.18
	
	\item \textbf{Elektromagnetischer Feldtensor}
	\begin{equation*}
	\vec{E} = - \vabla \Phi - \prt{\vec{A}}{t} \qquad \vec{B} = \vabla \times \vec{A}
	\end{equation*}
	\textbf{Feldstärketensor}
	\begin{align*}
	F^{\mu \nu} &\defeq \partial^{\mu} A^{\nu} - \partial^{\nu} A^{\mu}\\
	&= \prt{}{x_{\mu}} A^{\nu} - \prt{}{x_{\nu}} A^{\mu}
	\end{align*}
	$ \mu, \nu = 0, 1, 2, 3 $ ($ F $ ist also ein Tensor 2. Stufe oder hier eine $ 4 \times 4 $-Matrix)
	\begin{equation*}
	F^{\mu \nu} = - F^{\nu \mu}
	\end{equation*}
	\begin{equation*}
	\Rightarrow \quad F^{\mu \mu} = 0 \qquad \mu = 0, 1, 2, 3
	\end{equation*}
	\begin{equation*}
	\Rightarrow \quad 6 \tx{ unabhängige Elemente}
	\end{equation*}
	\begin{align*}
	F^{01} &= \prt{A^{1}}{x_0} - \prt{A^0}{x_1} = \frac{1}{c} \prt{A_x}{t} + \frac{1}{c} \prt{\Phi}{x}\\
	&= - \frac{1}{c} E_x = - F^{10}\\[5pt]
	F^{02} &= - \frac{1}{c} E_y = - F^{20} \qquad \quad F^{03} = - \frac{1}{c} E_z = - F^{30} \\[5pt]
	F^{12} &= \prt{A^2}{x_1} - \prt{A^1}{x_2} = - \prt{A_y}{x} + \prt{A_x}{y}\\
	&= - (\vabla \times \vec{A})_z = - B_z = - F^{21}\\[5pt]
	F^{13} &= B_y = - F^{31} \qquad \qquad \ \: F^{23} = - B_x = - F^{32}
	\end{align*}
	Hiermit und mit $ F_{\mu \nu} = g_{\mu \alpha} g_{\nu \beta} F^{\alpha \beta} $ erhalten wir:
	\begin{equation*}
	(F^{\mu \nu}) = \begin{pmatrix}
	0 & - \frac{1}{c} E_x & - \frac{1}{c} E_y & - \frac{1}{c} E_z \\[10pt]
	\frac{1}{c} E_x & 0 & - B_z & B_y \\[10pt]
	\frac{1}{c} E_y & B_z & 0 & - B_x \\[10pt]
	\frac{1}{c} E_z & - B_y & B_x & 0
	\end{pmatrix} \qquad \qquad
	(F_{\mu \nu}) = \begin{pmatrix}
	0 & \frac{1}{c} E_x & \frac{1}{c} E_y & \frac{1}{c} E_z \\[10pt]
	- \frac{1}{c} E_x & 0 & - B_z & B_y \\[10pt]
	- \frac{1}{c} E_y & B_z & 0 & - B_x \\[10pt]
	- \frac{1}{c} E_z & - B_y & B_x & 0
	\end{pmatrix}
	\end{equation*}
\end{enumerate}

\subsection{Kovariante Form der Maxwell-Gleichungen}

\frbox{Inhomogene Maxwell-Gleichungen}{\begin{equation*}
\partial_{\alpha} F^{\alpha \beta} = \mu_0 j^{\beta} \qquad \beta = 0, 1, 2, 3
\end{equation*}
}
\noindent
\textbf{$ \beta = 0 $ :}
\begin{equation*}
\mu_0 j^0 = \mu_0 c \rho = \sum_{\alpha} \prt{}{x^\alpha} F^{\alpha 0} = \frac{1}{c} \vabla \cdot \vec{E}
\end{equation*}
\begin{equation*}
\Rightarrow \quad \vabla \cdot \vec{E} = \mu_0 c^2 \rho = \frac{1}{\epsilon_0} \rho \quad \checkmark
\end{equation*}
\textbf{$ \beta = 1 $ :}
\begin{equation*}
\mu_0 j^1 = \mu_0 j_x = \sum_{\alpha} \prt{}{x^\alpha} F^{\alpha 1} = \frac{1}{c} \prt{}{t} \left(- \frac{1}{c} E_x\right) + \prt{B_z}{y} - \prt{B_y}{z} = - \frac{1}{c^2} \prt{E_x}{t} + (\vabla \times \vec{B})_x
\end{equation*}
Mit $ \beta = 2, 3 $ für $ y $- und $ z $-Komponenten ebenso:
\begin{equation*}
\Rightarrow \quad \vabla \times \vec{B} - \frac{1}{c^2} \prt{\vec{E}}{t} = \mu_0 \vec{j} \quad \checkmark
\end{equation*}
\textbf{Homogene Maxwell-Gleichungen}
\begin{equation*}
\rmbox{ \partial_{\lambda} F_{\mu \nu} + \partial_{\mu} F_{\nu \lambda} + \partial F_{\lambda \mu} = 0} \qquad \lambda, \mu, \nu = 0, 1, 2, 3
\end{equation*}
\noindent
Indizes zyklisch vertauscht!
\begin{equation*}
\vabla \cdot \vec{B} = 0 \qquad \vabla \times \vec{E} + \prt{\vec{B}}{t} = 0
\end{equation*}
Wegen der Antisymmetrie von $ F $ gibt es nur 4 unabhängige Gleichungen:
\begin{equation*}
\lambda, \mu, \nu \in \left\{ 123,\, 012,\, 013,\, 023 \right\}
\end{equation*}
und alle zyklischen Vertauschungen dieser Werte.\\[5pt]
\textbf{$ \lambda = \mu \neq \nu $}
\begin{equation*}
0 = \partial_{\lambda} F_{\lambda \nu} + \partial_{\lambda} \equalto{F_{\nu \lambda}}{- F_{\lambda \nu}} + \partial _{\nu} \equalto{F_{\lambda \lambda}}{0}
\end{equation*}
\begin{equation*}
\frac{1}{2} \epsilon^{\alpha \lambda \mu \nu} \partial_{\lambda} F_{\mu \nu} = 0 \qquad \alpha = 0, 1, 2, 3
\end{equation*}
\begin{equation*}
\epsilon^{\alpha \lambda \mu \nu} = \left\{ \begin{array}{r c}
+1 & \tx{ gerade Perm. von } 0123\\
-1 & \tx{ ungerade Perm. von } 0123\\
0 & \tx{ sonst }
\end{array} \right.
\end{equation*}
\textbf{$ \alpha = 0 $ :}
\begin{align*}
0 &= \frac{1}{2} \epsilon^{0 \lambda \mu \nu} \partial_{\lambda} F_{\mu \nu}\\
&= \frac{1}{2} \bigg(\equalto{\epsilon}{0}^{0123} \partial_1 F_{23} + \equalto{\epsilon}{-1}^{0132} \partial_{1} \equalto{F_{32}}{- F_{32}} + \equalto{\epsilon}{-1}^{0213} \partial_{2} \equalto{F_{13}}{- F_{31}} + \equalto{\epsilon}{1}^{0231} \partial_{2} F_{21} + \equalto{\epsilon}{-1}^{0312} \partial_{3} F_{12} + \equalto{\epsilon}{1}^{0321} \partial_{3} F_{21} \bigg)\\
&= \partial_1 F_{23} + \partial_2 F_{31} + \partial_3 F_{12}
\end{align*}
Definition \textbf{dualer Feldstärketensor}
\begin{equation*}
\tilde{F}^{\alpha \lambda} \defeq \frac{1}{2} \epsilon^{\alpha \lambda \mu \nu} F_{\mu \nu}
\end{equation*}
\frbox{Homogene Maxwell-Gleichungen}{
\begin{equation*}
\partial _{\lambda} \tilde{F}^{\alpha \lambda} = 0 \qquad \alpha = 0, 1, 2, 3
\end{equation*}
}
\noindent
\begin{align*}
\tilde{F}^{\alpha \alpha} &= \frac{1}{2} \epsilon^{\alpha \alpha \mu \nu} F_{\mu \nu} = 0\\[5pt]
\tilde{F}^{0 1} &= \frac{1}{2} \epsilon^{0 1 \mu \nu} F_{\mu \nu} \\
&= \frac{1}{2} \bigg( \equalto{\epsilon}{1}^{0123} F_{23} + \equalto{\epsilon}{-1}^{0132} \equalto{F_{32}}{- F_{23}} \bigg)\\
&= F_{23} = - B_x\\[5pt]
& \dots
\end{align*}
\begin{equation*}
(\tilde{F}^{\alpha \lambda}) = \begin{pmatrix}
0 & -B_x & - B_y & - B_z \\[10pt]
B_x & 0 & \frac{1}{c} E_z & -\frac{1}{c} E_y \\[10pt]
B_y & - \frac{1}{c} E_z & 0 & \frac{1}{c} E_x \\[10pt]
B_z & \frac{1}{c} E_y & - \frac{1}{c} E_x & 0
\end{pmatrix}
\end{equation*}
\textbf{$ \alpha = 0 $ :}
\begin{align*}
0 &= \partial _{\lambda} \tilde{F}^{\alpha \lambda} = \sum_{\lambda} \prt{}{x^{\lambda}} \tilde{F}^{0 \lambda} \\
&= - \prt{B_x}{x} - \prt{B_y}{y} - \prt{B_z}{z} = - \vabla \cdot \vec{B}\\[5pt]
\vabla \cdot \vec{B} &= 0 \quad \checkmark
\end{align*}
\textbf{$ \alpha = 1 $ :}
\begin{align*}
0 &= \sum_{\lambda} \prt{}{x^{\lambda}} \tilde{F}^{1 \lambda} \\
&= \frac{1}{c} \prt{}{t} B_x + \frac{1}{c} \prt{E_z}{y} - \frac{1}{c} \prt{E_y}{z} \\
&= \frac{1}{c} \left(\prt{B_x}{t} + (\vabla \times \vec{E})_x \right)\\
\end{align*}
mit $ \alpha = 2, 3 $ also die $ y $- und $ z $-Komponenten erhalten wir:
\begin{equation*}
\vabla \times \vec{E} + \prt{\vec{B}}{t} = 0 \quad \checkmark
\end{equation*}

\subsection{Lorentztransformation des Feldes}

$ S \to S' $
\begin{align*}
{F'}^{\mu \nu} &= \Lambda^{\mu}_{\ \alpha} \Lambda^{\nu}_{\ \beta} F^{\alpha \beta}\\
\tilde{F}^{\mu \nu} &= \Lambda^{\mu}_{\ \alpha} \Lambda^{\nu}_{\ \beta} \tilde{F}^{\alpha \beta}
\end{align*}
%
%
%
% T1
%
%
%
\begin{equation*}
(\Lambda^{\mu}_{\ \alpha}) = \begin{pmatrix}
\gamma & - \beta \gamma & 0 & 0 \\
- \beta \gamma & \gamma & 0 & 0 \\
0 & 0 & 1 & 0 \\
0 & 0 & 0 & 1
\end{pmatrix}
\end{equation*}
\begin{equation*}
F' = \Lambda F \Lambda
\end{equation*}
\begin{equation*}
\Rightarrow \quad (F'^{\mu \nu}) = \begin{pmatrix}
0 & - \frac{1}{c} E_x & \gamma (- \frac{1}{c} E_y + \beta B_z) & - \gamma (\frac{1}{c} E_z + \beta B_y) \\[10pt]
\frac{1}{c} E_x & 0 & \gamma (\frac{\beta}{1} E_y - B_z) & \gamma (\frac{\beta}{c} E_z + B_y) \\[10pt]
 - \gamma ( - \frac{1}{c} E_z + \beta B_z) & - \gamma (\frac{\beta}{c} E_y - B_y) & 0 & - B_x \\[10pt]
\gamma (\frac{1}{c} E_z + \beta B_y)  & - \gamma (\frac{\beta}{c} E_z + B_y) & B_x & 0
\end{pmatrix}
\end{equation*}
(zur Umformung benutzt: $ \frac{\gamma^2}{c} (\beta^2 - 1) = - \frac{1}{c} $)
\begin{equation*}
\Rightarrow \quad (F'_{\mu \nu}) = \begin{pmatrix}
0 & - \frac{1}{c} E'_x & - \frac{1}{c} E'_y & - \frac{1}{c} E'_z \\[10pt]
\frac{1}{c} E'_x & 0 & - B'_z & B'_y \\[10pt]
\frac{1}{c} E'_y & B'_z & 0 & - B'_x \\[10pt]
\frac{1}{c} E'_z & - B'_y & B'_x & 0
\end{pmatrix}
\end{equation*}
\begin{align*}
\Rightarrow \quad E'_x &= E_x \\[5pt]
E'_y &= - c \gamma (- \frac{1}{c} E_y + \beta B_z) \\
&= \gamma (E_y - v B_y) \\[5pt]
E'_z &= \gamma (E_z + v B_y) \\[10pt]
B'_x &= B_x \\
B'_y &= \gamma (B_y + \frac{v}{c^2} E_z) \\
B'_z &= \gamma (B_z - \frac{v}{c^2} E_y)
\end{align*}
Dies gilt nur für eine Geschwindigkeit $ v $ in $ x $-Richtung. Für eine allgemeine Geschwindigkeit gilt:
\begin{align*}
\vec{E}' &= \gamma (\vec{E} + \vec{v} \times \vec{B}) + (1 - \gamma) \frac{(\vec{E} \cdot \vec{v}) \vec{v}}{v^2} \\
\vec{B}' &= \gamma (\vec{B} - \frac{1}{c^2} \vec{v} \times \vec{E}) + (1 - \gamma) \frac{(\vec{B} \cdot \vec{v}) \vec{v}}{v^2}
\end{align*}
\emph{Beispiel:} Feld einer gleichförmig bewegten Punktladung\\[5pt]
Ladung: $ q $, Geschwindigkeit: $ \vec{v} = v \vec{e}_x $
%
%
%
% T2
%
%
%
\textbf{in $ S' $ :}
\begin{equation*}
\vec{E}' = \frac{q}{4 \pi \epsilon_0} \frac{\vec{r}'}{r'^3} \qquad \vec{B}' = 0
\end{equation*}
Das entspricht: $ S' \to S $ und einer Lorentztransformation (LT) $ - \vec{v} $:\\
\lcom{D.h. wir müssen in den zuvor hergeleiteten Gleichungen das gestrichene System vertauschen.}
\begin{align*}
E_x &= E_x' = \frac{q}{4 \pi \epsilon_0} \frac{x'}{r'^3} \\
E_y &= \gamma E_y' = \gamma \frac{q}{4 \pi \epsilon_0} \frac{y'}{r'^3} \\
E_z &= \gamma E_z' = \gamma \frac{q}{4 \pi \epsilon_0} \frac{z'}{r'^3} \\[5pt]
B_x &= B_x' = 0 \\
B_y &= - \gamma \frac{v}{c^2} E_z' = - \gamma \frac{v}{c^2} \frac{q}{4 \pi \epsilon_0} \frac{z'}{r'^3} \\
B_z &= \gamma \frac{v}{c^2} E_y' = \gamma \frac{v}{c^2} \frac{q}{4 \pi \epsilon_0} \frac{y'}{r'^3}
\end{align*}
\lcom{Dies sind die Felder im System $ S $ ausgedrückt in Koordinaten von $ S' $. Diese müssen wir nun noch Transformieren.}\\[5pt]
$ x', y', z' \to x, y, z, t $\\
$ x' = \gamma (x - vt) \quad y' = y \quad z' = z $\\
$ r'^3 = \left[\gamma^2 (x-vt)^2 + y^2 + z^2\right]^{\frac{3}{2}} $
\begin{equation*}
E_x = \frac{q}{4 \pi \epsilon_0} \frac{\gamma (x - vt)}{\left[ \dots \right]^{\frac{3}{2}}} \qquad \dots
\end{equation*}
Die Gleichungen haben also eine ähnliche Form.
\vspace{100pt}
\begin{center}
	\scalebox{10}[10]{$ \mathcal{FIN} $}
\end{center}
